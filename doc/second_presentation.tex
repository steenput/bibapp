\documentclass[10pt]{beamer}
\usepackage[utf8]{inputenc}
\usepackage[francais]{babel}
\usepackage[T1]{fontenc}
\usepackage[export]{adjustbox}
\newcommand\Fontvi{\fontsize{8}{7.2}\selectfont}

\usepackage{minted}
\usemintedstyle{colorful}
\usepackage{hyperref}
\hypersetup{
	colorlinks,
	citecolor=black,
	filecolor=black,
	linkcolor=black,
	urlcolor=blue
}

\usetheme{Frankfurt}
\usecolortheme{beaver}

\addtobeamertemplate{navigation symbols}{}{%
    \usebeamerfont{footline}%
    \usebeamercolor[fg]{footline}%
    \hspace{1em}%
    \insertframenumber/\inserttotalframenumber
}

\begin{document}
\logo{%
	\makebox[0.95\paperwidth]{%
		\includegraphics[width=2.5cm,keepaspectratio]{images/hepia.jpg}%
		\hfill%
		\includegraphics[width=2.5cm,keepaspectratio]{images/hesso.jpg}%
	}%
}

\title{BibApp Hepia}
\author{Steven Liatti}
\institute{Projet de semestre - Prof. Mickaël Hoerdt - Hepia ITI 3\up{ème} année}
\date{15 mars 2018}

\begin{frame}
\titlepage
\end{frame}

\begin{frame}
    \frametitle{Plan}
	\setcounter{tocdepth}{3}
	\tableofcontents
\end{frame}

\section{Introduction}
\subsection{Contexte}
\begin{frame}
	\frametitle{\secname}
	\framesubtitle{\subsecname}
    \begin{itemize}
        \item Bibliothèque aimerait devenir plus attractive (manque de fréquentation)
        \item Réseau NEBIS
        \item Nouveautés et périodiques
    \end{itemize}
    \Large\textbf{Besoin : ajouter du contenu personnalisé}
\end{frame}

% \subsection{BibApp V1}
% \begin{frame}
% 	\frametitle{\secname}
% 	\framesubtitle{\subsecname}
% 	\begin{columns}[T]
% 		\begin{column}{.65\textwidth}
%             \begin{columns}[T]
%                 \begin{column}{.3\textwidth}
%                     \begin{figure}
%                         \includegraphics[width=1\textwidth]{images/bibapp.png}
%                     \end{figure}
%                 \end{column}
%                 \begin{column}{.3\textwidth}
%                     \begin{figure}
%                         \includegraphics[width=1\textwidth]{images/bibapp2.png}
%                     \end{figure}
%                 \end{column}
%                 \begin{column}{.3\textwidth}
%                     \begin{figure}
%                         \includegraphics[width=1\textwidth]{images/bibapp3.png}
%                     \end{figure}
%                 \end{column}
%             \end{columns}
% 		\end{column}
% 		\begin{column}{.35\textwidth}
% 			\begin{itemize}
% 				\item Nouveautés listées par section
%                 \item Affiche les détails du livre
%                 \item Android et iOS
%                 \item Par Cicciù et Minelli
% 			\end{itemize}
% 		\end{column}
% 	\end{columns}
% \end{frame}

\subsection{Objectifs}
\begin{frame}
	\frametitle{\secname}
	\framesubtitle{\subsecname}
    \begin{itemize}
        \item Étude du framework Ionic (tuto + prototype)
        \item Architecture de l'application mobile/web (backend, frontend et interfaces)
        \item Réalisation de la base de données augmentée
        \begin{itemize}
            \item Basée sur les données de NEBIS
            \item Enrichie avec les données saisies par les bibliothécaires
            \item Accessible via le serveur REST avec mode client et administrateur
        \end{itemize}
        \item Design et implémentation des interfaces en collaboration avec les bibliothécaires
    \end{itemize}
\end{frame}

\section{Architecture}
\begin{frame}
	\frametitle{\secname}
	\begin{figure}
		\begin{center}
			\includegraphics[width=0.5\textwidth]{images/architecture_final.png}
		\end{center}
	\end{figure}
\end{frame}

\section{Réalisation}
\subsection{Wrapper}
\begin{frame}
	\frametitle{\secname}
	\framesubtitle{\subsecname}
	\begin{itemize}
        \item Serveur Node.js
        \item Appels à l'API RIB NEBIS
        \item Trois routes, retournant des données en JSON :
        \begin{itemize}
            \item Liste des nouveautés pour année et mois donnés
            \item Recherche de documents selon critères
            \item Obtention des infos d'un ouvrage par ISBN ou ISSN
        \end{itemize}
    \end{itemize}
\end{frame}

\subsection{Serveur}
\begin{frame}
	\frametitle{\secname}
	\framesubtitle{\subsecname}
	\begin{itemize}
        \item Serveur Node.js
    \end{itemize}
\end{frame}

\subsection{Base de données}
\begin{frame}
	\frametitle{\secname}
	\framesubtitle{\subsecname}
	\begin{itemize}
        \item Réalisée avec MongoDB et Mongoose
    \end{itemize}
\end{frame}

\subsection{Ionic}
\begin{frame}
	\frametitle{\secname}
	\framesubtitle{\subsecname}
	\begin{itemize}
        \item 
    \end{itemize}
\end{frame}

\section{Conclusion}
\begin{frame}
	\frametitle{\secname}
	\begin{itemize}
        \item Amélioration de mes compétences en développement web
        \item Découverte du framework Ionic
        \item Améliorations :
        \begin{itemize}
            \item Authentification côté Ionic
            \item Préférer une interface d'administration dédiée pour l'ajout de contenu
            \item Amélioration de l'expérience utilisateur sur l'app Ionic
        \end{itemize}
    \end{itemize}
\end{frame}

\end{document}
